% Make 2D illustrations of sPHENIX, sPHENIX-Forward and sPHENIX-EIC detectors

\documentclass[tikz]{standalone}
\usepackage{tikz}

\usepackage{tkz-euclide}
\usepackage[eulergreek]{sansmath}

\begin{document}
\title{EICDetector}

\definecolor{airforceblue}{rgb}{0.36, 0.54, 0.66}
\definecolor{bluegray}{rgb}{0.4, 0.6, 0.8}
\definecolor{amber(sae/ece)}{rgb}{1.0, 0.49, 0.0}
\definecolor{babyblueeyes}{rgb}{0.63, 0.79, 0.95}
\definecolor{bazaar}{rgb}{0.6, 0.47, 0.48}
\definecolor{blizzardblue}{rgb}{0.67, 0.9, 0.93}

\definecolor{tableau10_blue}{rgb}{0, 0.42, 0.64}
\definecolor{tableau10_orange}{rgb}{1, 0.50, 0.06}
\definecolor{tableau10_blue2}{rgb}{0.37, 0.62, 0.82}
\definecolor{tableau10_gray2}{rgb}{0.35,0.35,0.35}
\definecolor{tableau10_orange2}{rgb}{1, 0.74, 0.47}
\definecolor{tableau10_brown}{rgb}{0.78, 0.32, 0.0}

% zero-line	
\newcommand{ \beamcenter }{

	\draw [dashed, thin] ( -4.5,0 ) -- (4.5,0);
	
}


% draw line at fixed eta
\edef\etar{3} % maximum r for eta lines

\newcommand{ \etaline }[1]{
% calculate x-value to y value
\pgfmathparse{ \etar / tan( 2.0 * atan( exp( -1.0 * #1 ) )  )}
\edef\tmp{\pgfmathresult}
\draw [red,thick] ( 0,0 ) -- ( \tmp , \etar);
}

% BaBAr solenoid magnet		
\newcommand{\magnet}{

        % cryostat
	\draw[fill=tableau10_gray2] (-1.9250, 1.4200) -- (-1.9250, 1.8843) -- (-1.6290, 1.8843) -- (-1.6290, 1.7800) -- (1.6290, 1.7800) -- (1.6290, 1.8843) -- (1.9250, 1.8843) -- (1.9250, 1.4200) -- (-1.9250, 1.4200);                                                       
 
	%  coil
	% \draw[fill=amber(sae/ece)] (-1.7450, 1.5189) -- (1.7450, 1.5189) -- (1.7450, 1.4781) -- (-1.7450, 1.4781) -- (-1.7450, 1.5189);    
}	

\newcommand{\supportring}{

        	\draw[fill=tableau10_gray2, shift={(2025/1000,1370/1000)}, rotate around={0:(0,0)}] (0.0,0.0) rectangle (150/1000,580/1000);
        	\draw[fill=tableau10_gray2, shift={(-2025/1000-150/1000,1370/1000)}, rotate around={0:(0,0)}] (0.0,0.0) rectangle (150/1000,580/1000);
}	

% Barrel return yoke / HCAL
\newcommand{ \outerhcal }{

	% sPHENIX pCDR 2015, figure 10.4
	% start top-right corner
	\draw[fill=tableau10_blue] ( 3.1549 - 0.0320, 2.70 ) -- ( 3.1549 - 0.0320, 1.9500 ) -- ( 1.5420, 1.9500 ) -- ( 1.5420, 1.8200 ) -- ( -1.5420, 1.8200 ) --  ( -1.5420, 1.9500 )  -- ( -3.1549 + 0.0320, 1.9500 ) -- ( -3.1549 + 0.0320, 2.70 ) -- ( 3.1549 - 0.0320, 2.70 );
	
}

% Barrel Inner HCAL
\newcommand{ \innerhcal }{

	% sPHENIX pCDR 2015, figure 10.4
	% start top-right corner
	\draw[fill=tableau10_blue]  (1835.1/1000,1155.8/1000) -- (2175/1000,1370/1000) -- (-2175/1000,1370/1000) -- (-2175/1000,1155.8/1000) -- (1835.1/1000,1155.8/1000);
	
%	( 2.1750, 1.3700 ) -- ( 1.8305, 1.1350 ) -- ( -2.1750, 1.1350 ) -- ( -2.1750, 1.3700 ) -- ( 2.1750, 1.3700 );

}

% Return yoke e-side
\newcommand{ \edoor }{

	\draw[fill=tableau10_gray2, shift={(-3610/1000,300/1000)}, rotate around={0:(0,0)}] (0.0,0.0) rectangle (300/1000,2400/1000);
}
		
% Barrel ECAL
\newcommand{ \cecal }
{

	% sPHENIX pCDR 2015, figure 10.4
	% start top-right corner
	\draw[fill=tableau10_blue2] ( 1.4426, 1.0800 ) -- ( 1.4426, 0.9500 ) -- ( -1.4426, 0.9500 ) -- ( -1.4426, 1.0800 ) -- ( 1.4426, 1.0800 );

}

% Extended Barrel ECAL
\newcommand{ \cecalext }
{
	% start top-right corner
	%\draw[fill=tableau10_blue2] ( 1.727, 1.0800 ) -- ( 1.727, 1.0800 )  -- ( 1.429, 0.9500 ) -- ( -2.609, 0.9500 ) -- ( -2.609, 1.0800 ) -- ( 1.7275, 1.0800 );
	\draw[fill=tableau10_blue2] ( 1429/1000,900/1000) -- (1727.2/1000,1087.7/1000) -- (1727.2/1000,1155.8/1000) -- (-2599.6/1000,1155.8/1000) -- (-2599.6/1000,900/1000) --  ( 1429/1000,900/1000);

}

% TPC tracking
\newcommand{ \tpc }
{
	% fill in volume for TPC
	\draw[fill=tableau10_orange2, shift={(-1333.5/1000,200/1000)}, rotate around={0:(0,0)}] (0.0,0.0) rectangle (2667/1000,580/1000);

	% add Micromega layer
	\draw[fill=tableau10_orange2, shift={(-1100/1000,180/1000)}, rotate around={0:(0,0)}] (0.0,0.0) rectangle (2200/1000,20/1000);
}

% MAPS tracking
\newcommand{ \maps }
{
	\draw[fill=tableau10_orange2, shift={(172.5/1000,52/1000)}, rotate around={0:(0,0)}] (0.0,0.0) rectangle (-2172.5/1000,-30/1000);
}

%e-going GEM EIC
\newcommand{ \egem }
{
	%ETRK0
	\draw[fill=tableau10_orange, shift={(-205/1000,55/1000)}, rotate around={0:(0,0)}] (0.0,0.0) rectangle (30/1000,0.12);
	
	%ETRK1
	\draw[fill=tableau10_orange, shift={(-695/1000,55/1000)}, rotate around={0:(0,0)}] (0.0,0.0) rectangle (30/1000,0.12);
	
	%ETRK2
	\draw[fill=tableau10_orange, shift={(-1370/1000,55/1000)}, rotate around={0:(0,0)}] (0.0,0.0) rectangle (30/1000,730/1000);
	
	%ETRK3
	\draw[fill=tableau10_orange, shift={(-1600/1000,55/1000)}, rotate around={0:(0,0)}] (0.0,0.0) rectangle (30/1000,730/1000);
}

%e-going ECAL 
\newcommand{ \eecaleic }
{

	% ? cover eta -1.2 to -4.5
	% start top-right corner
	\draw[fill=tableau10_blue2, shift={(-1800/1000,55/1000)}, rotate around={0:(0,0)}] (0.0,0.0) rectangle (200/1000,730/1000);

}

%e-going Aerogel 
\newcommand{ \ericheic }
{
	% mRICH thickness: ~20 cm
	\draw[fill=tableau10_brown, shift={(-1570/1000,55/1000)}, rotate around={0:(0,0)}] (0.0,0.0) rectangle (200/1000,730/1000);

}
% Barrel DIRC
\newcommand{ \dirc }
{
	\draw[fill=tableau10_brown, shift={(1292/1000,890/1000)}, rotate around={0:(0,0)}] (0.0,0.0) rectangle (-3343.3/1000,-73/1000);
}

% h-going Aerogel
\newcommand{ \haerogel }
{

\draw[fill=tableau10_brown, shift={(2.66,0.83)}, rotate around={23.7:(0,0)}] (0.0,0.0) rectangle (0.2,0.77);

}

% h-going gas rich
\newcommand{ \hrichgas }
{

\draw [fill=tableau10_brown] (1.70078268,0.03000025073)  -- (2.693829981,0.02999998013) arc (-4.55909:43.78-4.55909:1.950) -- (1.419780399,0.8885501673) arc (43.0709:-6.82449:1.07087);

}

\newcommand{ \hgemeic }
{
	%FTRK0
	\draw[fill=tableau10_orange, shift={(0.175,0.025)}, rotate around={0:(0,0)}] (0.0,0.0) rectangle (0.03,0.15);
	
	%FTRK1
	\draw[fill=tableau10_orange, shift={(0.665,0.025)}, rotate around={0:(0,0)}] (0.0,0.0) rectangle (0.03,0.15);
	
	%FRRK2
	\draw[fill=tableau10_orange, shift={(1.340,49.65203779/1000)}, rotate around={0:(0,0)}] (0.0,0.0) rectangle (0.03,806.2556646/1000);
	
	%FTRK3
	\draw[fill=tableau10_orange, shift={(1.570,58.08005897/1000)}, rotate around={0:(0,0)}] (0.0,0.0) rectangle (0.03,378.9370362/1000);
	\draw[fill=tableau10_orange, shift={(1.570,58.08005897/1000+378.9370362/1000)}, rotate around={23.7:(0,0)}] (0.0,0.0) rectangle (0.03,490./1000);

	%FTRK4	
	\draw[fill=tableau10_orange, shift={(2.71,0.1)}, rotate around={0:(0,0)}] (0.0,0.0) rectangle (0.03,0.5115296);
	\draw[fill=tableau10_orange, shift={(2.71,0.1+0.5115296)}, rotate around={23.7:(0,0)}] (0.0,0.0) rectangle (0.03,0.95);
}

\newcommand{ \hecaleic }
{
	\draw[fill=tableau10_blue2, shift={(2868.5/1000,110/1000)}, rotate around={0:(0,0)}] (0.0,0.0) rectangle (430/1000,1810/1000);
}

\newcommand{ \hhcaleic }
{
	\draw[fill=tableau10_blue, shift={(3310/1000,50/1000)}, rotate around={0:(0,0)}] (0.0,0.0) rectangle (1000/1000,2650/1000);
}
	
%-------- EIC-sPHENIX--------------------------------------------------------------------------------
\begin{tikzpicture}[scale = 3,font=\sffamily\huge, label style/.style={font=\sffamily\huge}]

	\tkzInit[xmax=4.5,ymax=3,xmin=-4.5,ymin=-0.0]
	
    \tkzDrawX[label={z [m]}, line width=1]
    \tkzDrawY[label={y [m]}, line width=1]
    \tkzDefPoint(0,-0.1){A0}\tkzLabelPoint[below](A0){0}
    \tkzDefPoint(1,-0.1){A1}\tkzLabelPoint[below](A1){1}
    \tkzDefPoint(2,-0.1){A2}\tkzLabelPoint[below](A2){2}
    \tkzDefPoint(3,-0.1){A3}\tkzLabelPoint[below](A3){3}
    \tkzDefPoint(4,-0.1){A4}\tkzLabelPoint[below](A4){4}
    \tkzDefPoint(-0.1,3){B3}\tkzLabelPoint[left](B3){3}
    	
	%\beamcenter
	\supportring
	\magnet
	\outerhcal
	\edoor
	\innerhcal
	%\cecal
	\cecalext
	\hgemeic
	\hecaleic
	\hhcaleic
	\tpc
	\maps			
	\dirc	
	\hrichgas
	\haerogel
	\egem
	\eecaleic
	\ericheic
	
	\draw[fill=tableau10_gray2, shift={(-4,3.3+0.1)}] (0.0,0.0) rectangle (0.3,0.15);
	\draw[fill=tableau10_blue, shift={(-4,3.0+0.1)}] (0.0,0.0) rectangle (0.3,0.15);
	\draw[fill=tableau10_blue2, shift={(-4,2.7+0.1)}] (0.0,0.0) rectangle (0.3,0.15);

	\node[] at (-2.2,3.375+0.1) {Magnet, support, flux return};
	\node[] at (-2.64,3.075+0.1) {Hadron calorimeter};
	\node[] at (-2.22,2.775+0.1) {Electromagnetic calorimeter};

	\draw[fill=tableau10_orange2, shift={(2,3.3+0.1)}] (0.0,0.0) rectangle (0.3,0.15);
	\draw[fill=tableau10_orange, shift={(2,3.0+0.1)}] (0.0,0.0) rectangle (0.3,0.15);
	\draw[fill=tableau10_brown, shift={(2,2.7+0.1)}] (0.0,0.0) rectangle (0.3,0.15);

	\node[] at (3.13,3.375+0.1) {TPC + MAPS};
	\node[] at (2.7,3.075+0.1) {GEM};
	\node[] at (2.7,2.775+0.1) {RICH};


\end{tikzpicture}

\end{document}